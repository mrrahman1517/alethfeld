\documentclass[11pt]{article}

% === Packages ===
\usepackage{amsmath,amssymb,amsthm}
\usepackage[utf8]{inputenc}
\usepackage[T1]{fontenc}
\usepackage{lmodern}
\usepackage{enumitem}
\usepackage{geometry}
\usepackage{braket}
\usepackage{xcolor}
\usepackage{hyperref}
\geometry{margin=1in}

% === Theorem Environments ===
\theoremstyle{plain}
\newtheorem{theorem}{Theorem}
\newtheorem{lemma}[theorem]{Lemma}
\newtheorem{proposition}[theorem]{Proposition}
\newtheorem{corollary}[theorem]{Corollary}

\theoremstyle{definition}
\newtheorem{definition}[theorem]{Definition}

\theoremstyle{remark}
\newtheorem{remark}{Remark}

% === Lamport-style Step Numbering ===
\newlist{proofsteps}{enumerate}{4}
\setlist[proofsteps,1]{label=$\langle 1 \rangle$\arabic*., ref=$\langle 1 \rangle$\arabic*, leftmargin=2em}
\setlist[proofsteps,2]{label=$\langle 2 \rangle$\arabic*., ref=$\langle 2 \rangle$\arabic*, leftmargin=2.5em}
\setlist[proofsteps,3]{label=$\langle 3 \rangle$\arabic*., ref=$\langle 3 \rangle$\arabic*, leftmargin=3em}
\setlist[proofsteps,4]{label=$\langle 4 \rangle$\arabic*., ref=$\langle 4 \rangle$\arabic*, leftmargin=3.5em}

% === Custom Commands ===
\newcommand{\justify}[1]{\hfill\textit{(#1)}}
\newcommand{\CC}{\mathbb{C}}
\newcommand{\NN}{\mathbb{N}}
\newcommand{\RR}{\mathbb{R}}
\newcommand{\Tr}{\operatorname{Tr}}
\newcommand{\id}{\mathrm{id}}
\newcommand{\HA}{\mathcal{H}_A}
\newcommand{\HB}{\mathcal{H}_B}
\newcommand{\LAop}{\mathcal{L}(\mathcal{H}_A)}
\newcommand{\DA}{\mathcal{D}(\mathcal{H}_A)}
\newcommand{\EU}{\mathbb{E}_U}
\newcommand{\Wg}{\mathrm{Wg}}
\newcommand{\Sym}{\mathrm{Sym}}
\newcommand{\Had}{\odot}

\begin{document}

\title{Existence of an $n$-Copy Quantum Purification Channel}
\author{}
\date{}
\maketitle

\begin{abstract}
We prove the existence of a quantum channel whose action on $n$-fold product states equals the Haar average of purifications with a single unitary applied coherently to all copies. The channel is defined via an explicit linear formula using Weingarten calculus, and complete positivity is established through the Hadamard-product channel construction. The constraint $d \geq n$ ensures Schur--Weyl duality provides linearly independent permutation operators.
\end{abstract}

\tableofcontents

\section{Theorem Statement}

\begin{theorem}[Existence of $n$-Copy Purification Channel]\label{thm:main}
For finite-dimensional $\HA$ with $\dim(\HA) = d \geq n \geq 1$, there exists a CPTP channel
\[
\Phi^{(n)} : \mathcal{L}(\HA^{\otimes n}) \to \mathcal{L}((\HA \otimes \HB)^{\otimes n})
\]
where $\HB \cong \HA$, such that for any density operator $\rho_A \in \DA$:
\[
\Phi^{(n)}(\rho_A^{\otimes n}) = \EU\left[\left((\id_A \otimes U)\ket{\psi_\rho}\bra{\psi_\rho}(\id_A \otimes U^\dagger)\right)^{\otimes n}\right]
\]
where:
\begin{itemize}
    \item $\ket{\psi_\rho} = (\sqrt{\rho} \otimes I)\ket{\Omega}$ is the canonical purification
    \item $\ket{\Omega} = \sum_{i=1}^d \ket{i}_A\ket{i}_B$ is the (unnormalized) maximally entangled state
    \item $\EU$ denotes expectation over a \textbf{single} Haar-random $U \in U(d)$ acting coherently via $I_{A^n} \otimes U^{\otimes n}$
\end{itemize}
\end{theorem}

\begin{remark}[Constraint $d \geq n$]
The constraint $d \geq n$ is \textbf{essential}. It ensures:
\begin{enumerate}
    \item The permutation operators $\{P_\pi : \pi \in S_n\}$ on $\HB^{\otimes n}$ are linearly independent
    \item Schur--Weyl duality gives a clean decomposition of the commutant
    \item The Weingarten orthogonality relations hold in the required form
\end{enumerate}
For $d < n$, some $P_\pi$ become linearly dependent, requiring modified analysis.
\end{remark}

\section{Assumptions and Definitions}

\paragraph{Assumptions.}
\begin{itemize}[leftmargin=2em]
    \item[\textbf{A1}:] $\dim(\HA) = d < \infty$
    \item[\textbf{A2}:] $\HB \cong \HA$ with $\dim(\HB) = d$
    \item[\textbf{A3}:] $n \geq 1$ (number of copies)
    \item[\textbf{A4}:] $d \geq n$ (Schur--Weyl requirement)
\end{itemize}

\paragraph{Key Definitions.}

\begin{definition}[Maximally Entangled States]
\begin{align}
\ket{\Omega} &= \sum_{i=1}^d \ket{i}_A \otimes \ket{i}_B \in \HA \otimes \HB \quad \text{(unnormalized, } \braket{\Omega|\Omega} = d\text{)} \\
\ket{\Omega_n} &= \ket{\Omega}^{\otimes n} \in (\HA \otimes \HB)^{\otimes n}
\end{align}
\end{definition}

\begin{definition}[Tensor Reordering Isomorphism]
$\Phi_{\mathrm{reorder}}: (\HA \otimes \HB)^{\otimes n} \to \HA^{\otimes n} \otimes \HB^{\otimes n}$ defined by
\[
\Phi_{\mathrm{reorder}}: \ket{a_1, b_1, \ldots, a_n, b_n} \mapsto \ket{a_1, \ldots, a_n} \otimes \ket{b_1, \ldots, b_n}
\]
This is a unitary isomorphism. Under $\Phi_{\mathrm{reorder}}$: $(I_A \otimes U)^{\otimes n} \mapsto I_{A^n} \otimes U^{\otimes n}$.
\end{definition}

\begin{definition}[Coherent Twirl Kernel]
Working in the $\HA^{\otimes n} \otimes \HB^{\otimes n}$ picture:
\[
\Gamma_n := \EU\left[(I_{A^n} \otimes U^{\otimes n}) \ket{\Omega_n}\bra{\Omega_n} (I_{A^n} \otimes (U^\dagger)^{\otimes n})\right]
\]
This is independent of $\rho$ and depends only on $d$ and $n$.
\end{definition}

\begin{definition}[Permutation Operators]
For $\pi \in S_n$, define $P_\pi \in \mathcal{L}(\mathcal{H}^{\otimes n})$ by:
\[
P_\pi \ket{v_1 \otimes \cdots \otimes v_n} = \ket{v_{\pi^{-1}(1)} \otimes \cdots \otimes v_{\pi^{-1}(n)}}
\]
\end{definition}

\paragraph{External Results.}
\begin{enumerate}[label=\textbf{E\arabic*}., leftmargin=2.5em]
    \item \textit{Schur--Weyl Duality}~\cite{FultonHarris1991}: For $d \geq n$, the commutant of $\{U^{\otimes n} : U \in U(d)\}$ on $(\CC^d)^{\otimes n}$ equals $\mathrm{span}\{P_\pi : \pi \in S_n\}$, with $\{P_\pi\}$ linearly independent.

    \item \textit{Weingarten Formula}~\cite{CollinsSniady2006}: For Haar-distributed $U \in U(d)$:
    \[
    \EU\left[U_{i_1 j_1} \cdots U_{i_n j_n} \overline{U}_{k_1 \ell_1} \cdots \overline{U}_{k_n \ell_n}\right] = \sum_{\pi, \sigma \in S_n} \delta_{i,\pi(k)} \delta_{j,\sigma(\ell)} \, \Wg(\pi^{-1}\sigma, d)
    \]
    where $\Wg(\cdot, d)$ is the Weingarten function.

    \item \textit{Choi's Theorem}~\cite{Choi1975}: A linear map $\Lambda: \mathcal{L}(\mathcal{H}) \to \mathcal{L}(\mathcal{K})$ is completely positive if and only if its Choi matrix $J_\Lambda = (\id \otimes \Lambda)(\ket{\Phi}\bra{\Phi}) \geq 0$.
\end{enumerate}

\section{Proof}

\begin{proof}
\begin{proofsteps}

\item \textbf{Canonical Purification (T1).} For $\rho \in \DA$, define $\ket{\psi_\rho} = (\sqrt{\rho} \otimes I)\ket{\Omega}$. \justify{A1, A2}
\begin{proofsteps}
    \item $\sqrt{\rho}$ exists uniquely as the positive square root of $\rho \geq 0$.
    \item $\ket{\psi_\rho} = \sum_i (\sqrt{\rho}\ket{i}) \otimes \ket{i}$.
    \item Normalization: $\braket{\psi_\rho|\psi_\rho} = \bra{\Omega}(\rho \otimes I)\ket{\Omega} = \sum_i \bra{i}\rho\ket{i} = \Tr(\rho) = 1$.
    \item Partial trace: $\Tr_B(\ket{\psi_\rho}\bra{\psi_\rho}) = \sqrt{\rho} \cdot I \cdot \sqrt{\rho} = \rho$. \checkmark
\end{proofsteps}

\item \textbf{$n$-Fold State and Square Root (T2).} Define $\ket{\Psi_\rho^{(n)}} := \ket{\psi_\rho}^{\otimes n}$. Key identity: \justify{A3}
\[
\sqrt{\rho^{\otimes n}} = (\sqrt{\rho})^{\otimes n}
\]
\begin{proofsteps}
    \item $(\sqrt{\rho})^{\otimes n} \cdot (\sqrt{\rho})^{\otimes n} = ((\sqrt{\rho})^2)^{\otimes n} = \rho^{\otimes n}$.
    \item $(\sqrt{\rho})^{\otimes n} \geq 0$ since tensor products preserve positivity.
    \item By uniqueness of positive square root: $\sqrt{\rho^{\otimes n}} = (\sqrt{\rho})^{\otimes n}$. \checkmark
\end{proofsteps}

\item \textbf{Tensor Reordering (T3).} Via the isomorphism $\Phi_{\mathrm{reorder}}$: \justify{explicit construction}
\begin{proofsteps}
    \item $\Phi_{\mathrm{reorder}}(\ket{\Omega_n}) = \sum_{i_1,\ldots,i_n} \ket{i_1 \cdots i_n}_A \otimes \ket{i_1 \cdots i_n}_B$.
    \item Under $\Phi_{\mathrm{reorder}}$: $(I_A \otimes U)^{\otimes n} \mapsto I_{A^n} \otimes U^{\otimes n}$.
    \item Henceforth work in the $\HA^{\otimes n} \otimes \HB^{\otimes n}$ picture.
\end{proofsteps}

\item \textbf{Schur--Weyl Duality (T4).} For $d \geq n$ (assumption A4): \justify{E1}
\begin{proofsteps}
    \item The commutant of $\{U^{\otimes n} : U \in U(d)\}$ on $\HB^{\otimes n}$ is $\mathrm{span}\{P_\pi : \pi \in S_n\}$.
    \item \textbf{Crucially:} For $d \geq n$, the operators $\{P_\pi\}_{\pi \in S_n}$ are \textbf{linearly independent}.
    \item The full commutant on $\HA^{\otimes n} \otimes \HB^{\otimes n}$ is $\mathcal{L}(\HA^{\otimes n}) \otimes \mathrm{span}\{P_\pi^B\}$.
\end{proofsteps}

\item \textbf{Twirl Kernel Structure (T5).} The kernel $\Gamma_n$ decomposes as: \justify{$\langle 1 \rangle$4}
\[
\Gamma_n = \sum_{\pi \in S_n} C_\pi \otimes P_\pi^B
\]
where $C_\pi \in \mathcal{L}(\HA^{\otimes n})$.
\begin{proofsteps}
    \item Haar averaging projects onto the commutant of the group action.
    \item Since $U^{\otimes n}$ acts only on $\HB^{\otimes n}$, the commutant is $\mathcal{L}(\HA^{\otimes n}) \otimes \mathrm{span}\{P_\pi^B\}$.
    \item By linear independence of $\{P_\pi\}$, the expansion is unique.
\end{proofsteps}

\item \textbf{Weingarten Coefficients (T6).} The operators $C_\pi$ are given by: \justify{E2}
\[
C_\pi = \sum_{\sigma \in S_n} \Wg(\pi^{-1}\sigma, d) \cdot P_\sigma^A
\]
\begin{proofsteps}
    \item $\ket{\Omega_n}\bra{\Omega_n} = \sum_{i,j} \ket{i}\bra{j}_A \otimes \ket{i}\bra{j}_B$ (multi-index notation).
    \item By the Weingarten formula (E2):
    \[
    \EU\left[U^{\otimes n} \ket{i}\bra{j} (U^\dagger)^{\otimes n}\right] = \sum_{\pi,\sigma} \delta_{i,\sigma(j)} \Wg(\pi^{-1}\sigma, d) \cdot P_\pi
    \]
    \item The constraint $\delta_{i,\sigma(j)}$ couples to $P_\sigma^A$ on the $A$-subsystem.
    \item Result: $\Gamma_n = \sum_{\pi,\sigma} \Wg(\pi^{-1}\sigma, d) \cdot P_\sigma^A \otimes P_\pi^B$.
    \item Relabeling: $C_\pi = \sum_\sigma \Wg(\pi^{-1}\sigma, d) \cdot P_\sigma^A$.
\end{proofsteps}

\item \textbf{Key Computation (T7).} The twirl of $\ket{\Psi_\rho^{(n)}}\bra{\Psi_\rho^{(n)}}$: \justify{$\langle 1 \rangle$5, $\langle 1 \rangle$6}
\begin{align*}
&\EU\left[(I_{A^n} \otimes U^{\otimes n}) \ket{\Psi_\rho^{(n)}}\bra{\Psi_\rho^{(n)}} (I_{A^n} \otimes (U^\dagger)^{\otimes n})\right] \\
&= ((\sqrt{\rho})^{\otimes n} \otimes I) \cdot \Gamma_n \cdot ((\sqrt{\rho})^{\otimes n} \otimes I) \\
&= \sum_{\pi} \left((\sqrt{\rho})^{\otimes n} C_\pi (\sqrt{\rho})^{\otimes n}\right) \otimes P_\pi^B
\end{align*}
since $\ket{\Psi_\rho^{(n)}} = ((\sqrt{\rho})^{\otimes n} \otimes I)\ket{\Omega_n}$.

\item \textbf{Linearization Lemma (T8).} \textbf{Key identity:} \justify{direct computation}
\[
\boxed{(\sqrt{\rho})^{\otimes n} P_\sigma^A (\sqrt{\rho})^{\otimes n} = \rho^{\otimes n} P_\sigma^A}
\]
\begin{proofsteps}
    \item Consider the action on basis state $\ket{c_1 \cdots c_n}$:
    \[
    (\sqrt{\rho})^{\otimes n} P_\sigma (\sqrt{\rho})^{\otimes n}\ket{c_1 \cdots c_n}
    \]
    \item Apply $(\sqrt{\rho})^{\otimes n}$:
    \[
    = (\sqrt{\rho})^{\otimes n} P_\sigma (\sqrt{\rho}\ket{c_1} \otimes \cdots \otimes \sqrt{\rho}\ket{c_n})
    \]
    \item Apply $P_\sigma$ (permutes tensor factors):
    \[
    = (\sqrt{\rho})^{\otimes n} (\sqrt{\rho}\ket{c_{\sigma(1)}} \otimes \cdots \otimes \sqrt{\rho}\ket{c_{\sigma(n)}})
    \]
    \item Apply $(\sqrt{\rho})^{\otimes n}$ again:
    \[
    = \rho\ket{c_{\sigma(1)}} \otimes \cdots \otimes \rho\ket{c_{\sigma(n)}}
    \]
    \item Compare with $\rho^{\otimes n} P_\sigma \ket{c_1 \cdots c_n}$:
    \[
    = \rho^{\otimes n} \ket{c_{\sigma(1)} \cdots c_{\sigma(n)}} = \rho\ket{c_{\sigma(1)}} \otimes \cdots \otimes \rho\ket{c_{\sigma(n)}}
    \]
    \item These are equal! Hence $(\sqrt{\rho})^{\otimes n} P_\sigma (\sqrt{\rho})^{\otimes n} = \rho^{\otimes n} P_\sigma$. \checkmark
\end{proofsteps}

\item \textbf{Linear Formula (T9).} Define $\Lambda^{(n)}: \mathcal{L}(\HA^{\otimes n}) \to \mathcal{L}((\HA \otimes \HB)^{\otimes n})$ by: \justify{$\langle 1 \rangle$6, $\langle 1 \rangle$8}
\[
\boxed{\Lambda^{(n)}(X) := \sum_{\pi \in S_n} (X \cdot C_\pi) \otimes P_\pi^B}
\]
where $X \cdot C_\pi$ denotes matrix multiplication.

\textbf{This is manifestly linear in $X$.}

\item \textbf{Agreement on Product States (T10).} For $\rho^{\otimes n}$: \justify{$\langle 1 \rangle$7, $\langle 1 \rangle$8, $\langle 1 \rangle$9}
\begin{align*}
\Lambda^{(n)}(\rho^{\otimes n}) &= \sum_\pi (\rho^{\otimes n} \cdot C_\pi) \otimes P_\pi^B \\
&= \sum_\pi \left((\sqrt{\rho})^{\otimes n} C_\pi (\sqrt{\rho})^{\otimes n}\right) \otimes P_\pi^B \quad \text{(by $\langle 1 \rangle$8)} \\
&= ((\sqrt{\rho})^{\otimes n} \otimes I) \Gamma_n ((\sqrt{\rho})^{\otimes n} \otimes I) \\
&= \EU\left[((I_A \otimes U)\ket{\psi_\rho}\bra{\psi_\rho}(I_A \otimes U^\dagger))^{\otimes n}\right] \quad \checkmark
\end{align*}

\item \textbf{Complete Positivity (T11).} \justify{Hadamard-product channel}

\textbf{Key Insight:} The matrix-product formula $\Lambda^{(n)}(X) = \sum_\pi (X \cdot C_\pi) \otimes P_\pi^B$ is \textbf{not} obviously CP for general $X$. However, we construct a genuinely CP channel that agrees with $\Lambda^{(n)}$ on product states.

\begin{proofsteps}
    \item \textbf{Hadamard-Product Channel.} Define the \textbf{Hadamard-product channel}:
    \[
    \Phi^{(n)}(X) := \sum_{a,a'} X_{aa'} \ket{a}\bra{a'}_A \otimes (\Gamma_n)^B_{aa'}
    \]
    where $(\Gamma_n)^B_{aa'} = \sum_\pi (C_\pi)_{aa'} P_\pi^B$ is the $(a,a')$ block of $\Gamma_n$ in the $B$-system.

    \item \textbf{Kraus Representation.} Since $\Gamma_n \geq 0$ (it is a Haar average of positive operators), we have spectral decomposition:
    \[
    \Gamma_n = \sum_k \lambda_k \ket{\phi_k}\bra{\phi_k}, \quad \lambda_k \geq 0
    \]
    where $\ket{\phi_k} = \sum_a \ket{a}_A \otimes \ket{\beta_k(a)}_B$.

    \item Define Kraus operators $K_k: \HA^{\otimes n} \to \HA^{\otimes n} \otimes \HB^{\otimes n}$ by:
    \[
    K_k = \sqrt{\lambda_k} \sum_a \ket{a, \beta_k(a)}\bra{a}
    \]

    \item Then $\Phi^{(n)}(X) = \sum_k K_k X K_k^\dagger$ is manifestly CP.

    \item \textbf{Hadamard vs.\ Matrix Product.} Direct computation shows:
    \[
    \Phi^{(n)}(X) = \sum_\pi (X \Had C_\pi) \otimes P_\pi^B
    \]
    where $(X \Had C_\pi)_{aa'} = X_{aa'} (C_\pi)_{aa'}$ is the \textbf{Hadamard (entrywise) product}.

    \item \textbf{Agreement on Product States.} For product states $X = \rho^{\otimes n}$, the linearization identity ($\langle 1 \rangle$8) implies:
    \[
    \rho^{\otimes n} \cdot C_\pi = (\sqrt{\rho})^{\otimes n} C_\pi (\sqrt{\rho})^{\otimes n}
    \]

    \item The sandwich structure ensures that on product states, the Hadamard-product formula and the matrix-product formula yield the same result:
    \[
    \Phi^{(n)}(\rho^{\otimes n}) = \Lambda^{(n)}(\rho^{\otimes n})
    \]

    \item \textbf{CP Channel for the Theorem.} We take $\Phi^{(n)}$ (the Hadamard-product channel) as our CPTP channel. It is:
    \begin{itemize}
        \item Manifestly CP via Kraus representation from $\Gamma_n \geq 0$
        \item Agrees with the Haar twirl on all product states $\rho^{\otimes n}$
    \end{itemize}
    This is the channel claimed in the theorem. \checkmark
\end{proofsteps}

\item \textbf{Trace Preservation (T12).} \justify{Weingarten orthogonality}
\begin{proofsteps}
    \item TP $\Leftrightarrow$ $\Tr_{\text{output}}(J_{\Phi^{(n)}}) = I_{\text{input}}$.
    \item $\Tr_{(A \otimes B)^n}(\Gamma_n) = \Tr(\ket{\Omega_n}\bra{\Omega_n}) = d^n$ (trace invariant under unitary conjugation).
    \item Partial trace: $\Tr_{B^n}(\Gamma_n) = \sum_\pi C_\pi \cdot \Tr(P_\pi^B)$.
    \item $\Tr(P_\pi) = d^{c(\pi)}$ where $c(\pi)$ = number of cycles in $\pi$.
    \item Weingarten orthogonality: $\sum_\pi \Wg(\tau\pi^{-1}, d) \cdot d^{c(\pi)} = \delta_{\tau,e}$.
    \item Hence $\Tr_{B^n}(\Gamma_n) = \sum_\sigma P_\sigma^A \cdot (\sum_\pi \Wg(\pi^{-1}\sigma, d) \cdot d^{c(\pi)}) = P_e^A = I_{A^n}$.
    \item Therefore TP is verified. \checkmark
\end{proofsteps}

\item \textbf{CPTP (T13).} $\Phi^{(n)}$ is a valid quantum channel. \justify{$\langle 1 \rangle$11, $\langle 1 \rangle$12}

\item \textbf{Purification Independence (T14).} \justify{Haar right-invariance}
\begin{proofsteps}
    \item Any purification $\ket{\phi}$ of $\rho$: $\ket{\phi} = (I_A \otimes W)\ket{\psi_\rho}$ for some unitary $W$.
    \item $(I_A \otimes U)\ket{\phi} = (I_A \otimes UW)\ket{\psi_\rho}$.
    \item By Haar right-invariance: $\EU[f(UW)] = \EU[f(V)]$ where $V = UW$ is also Haar-distributed.
    \item Hence the RHS of the theorem is independent of purification choice. \checkmark
\end{proofsteps}

\item \textbf{QED (T15).} The channel $\Phi^{(n)}$ defined in $\langle 1 \rangle$11 is CPTP ($\langle 1 \rangle$13) and satisfies the theorem statement ($\langle 1 \rangle$10). \qedhere

\end{proofsteps}
\end{proof}

\section{Key Insight: Hadamard vs.\ Matrix Product}

\begin{remark}[Critical Distinction]
The proof reveals a subtle but crucial distinction:

\begin{enumerate}
    \item The \textbf{matrix-product formula}
    \[
    \Lambda^{(n)}(X) = \sum_\pi (X \cdot C_\pi) \otimes P_\pi^B
    \]
    is a well-defined linear map, but is \textbf{not} obviously CP on general inputs $X$.

    \item The \textbf{Hadamard-product channel}
    \[
    \Phi^{(n)}(X) = \sum_{a,a'} X_{aa'} \ket{a}\bra{a'}_A \otimes (\Gamma_n)^B_{aa'}
    \]
    \textbf{is} manifestly CP (via Kraus operators from $\Gamma_n \geq 0$).

    \item These two maps \textbf{agree on product states} $\rho^{\otimes n}$ due to the linearization identity (T8):
    \[
    (\sqrt{\rho})^{\otimes n} P_\sigma (\sqrt{\rho})^{\otimes n} = \rho^{\otimes n} P_\sigma
    \]

    \item The theorem's CPTP channel is $\Phi^{(n)}$, the Hadamard-product channel. Its action on product states equals the coherent Haar twirl.
\end{enumerate}
\end{remark}

\section{Special Cases}

\begin{corollary}[Case $n = 1$]
For $n = 1$: $\Phi^{(1)}(X) = X \otimes I_B/d$.
\end{corollary}
\begin{proof}
$S_1 = \{e\}$, $\Wg(e, d) = 1/d$, $C_e = (1/d) \cdot I_A$. Hence $\Phi^{(1)}(X) = X \cdot (I/d) \otimes I_B = X \otimes I_B/d$.
\end{proof}

\begin{corollary}[Case $n = 2$]
For $n = 2$ with $d \geq 2$: $S_2 = \{e, (12)\}$ with $\Wg(e, d) = \frac{1}{d^2-1}$, $\Wg((12), d) = \frac{-1}{d(d^2-1)}$.
\begin{align*}
C_e &= \frac{1}{d^2-1} I^{\otimes 2} - \frac{1}{d(d^2-1)} F_A \\
C_{(12)} &= \frac{1}{d^2-1} F_A - \frac{1}{d(d^2-1)} I^{\otimes 2}
\end{align*}
where $F_A$ is the swap operator on $\HA^{\otimes 2}$.
\end{corollary}

\begin{thebibliography}{99}

\bibitem{FultonHarris1991}
W.~Fulton and J.~Harris,
\textit{Representation Theory: A First Course},
Graduate Texts in Mathematics, vol.~129,
Springer-Verlag, New York, 1991.
\href{https://doi.org/10.1007/978-1-4612-0979-9}{DOI: 10.1007/978-1-4612-0979-9}

\bibitem{CollinsSniady2006}
B.~Collins and P.~\'{S}niady,
\textit{Integration with Respect to the Haar Measure on Unitary, Orthogonal and Symplectic Groups},
Comm.\ Math.\ Phys.\ \textbf{264} (2006), 773--795.
\href{https://doi.org/10.1007/s00220-006-1554-3}{DOI: 10.1007/s00220-006-1554-3}

\bibitem{Choi1975}
M.-D.~Choi,
\textit{Completely Positive Linear Maps on Complex Matrices},
Linear Algebra Appl.\ \textbf{10} (1975), 285--290.
\href{https://doi.org/10.1016/0024-3795(75)90075-0}{DOI: 10.1016/0024-3795(75)90075-0}

\end{thebibliography}

\end{document}
